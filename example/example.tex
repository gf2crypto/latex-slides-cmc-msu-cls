%! Author = Ivan Chizhov
%! Date = 08.11.2022

% Preamble
\documentclass{../cmcbeamer}
\uselanguage{Russian}
\languagepath{Russian}
\usepackage{blindtext}

\graphicspath{{images/}{../images/}}

\title{Пример использования презентации класса `cmcbeamer`}
\subtitle{Курсовая работа}
\author{%
  Петров Алексей Иванович\\[\baselineskip]%\texorpdfstring{\\[\baselineskip]}{}
  \makebox[6cm]{}
  \parbox{0.3\textwidth}{
    \fontsize{8}{10}\selectfont
    \textit{%
      Науч. рук.:\hfill \break%
      Чижов Иван Владимирович,\hfill\break%
      доцент кафедры ИБ ВМК МГУ,\hfill\break%
      канд. физ.-мат. наук\hfill}
  }
}

\institute[ВМК МГУ]{%
  МГУ имени М.В.Ломоносова\\
  факультет вычислительной математики и кибернетики\\
  кафедра информационной безопасности
}

\date[]{\today}

\subject{Доклад}


% Document
\begin{document}

\section{Первый раздел}\label{sec:example:first}


\begin{frame}\frametitle{Заголовок слайда}
  This is some text in the first frame. This is some text in the first
  frame. This is some text in the first frame.

  \textit{Курсивный шрифт}.

  \textbf{Жирный шрифт}.

  \textit{\textbf{Курсивный жирный шрифт}}.

\end{frame}

\begin{frame}\frametitle{Sample frame title}
  This is a text in second frame.  For the sake of showing an example.

  \begin{itemize}
  \item<1-> Text visible on slide 1
  \item<2-> Text visible on slide 2
  \item<3> Text visible on slide 3
  \item<4-> Text visible on slide 4
  \end{itemize}
\end{frame}

\begin{frame}
  \frametitle{There Is No Largest Prime Number}
  \framesubtitle{The proof uses \textit{reductio ad absurdum}.}

  \begin{theorem}
    There is no largest prime number.
  \end{theorem}
  
  \begin{proof}
    \begin{enumerate}
    \item<1-| alert@1> Suppose $p$ were the largest prime number.
    \item<2-> Let $q$ be the product of the first $p$ numbers.
    \item<3-> Then $q+1$ is not divisible by any of them.
    \item<1-> Thus $q+1$ is also prime and greater than $p$.\qedhere
    \end{enumerate}
  \end{proof}
\end{frame}

\begin{frame}
\frametitle{Two-column slide}
\begin{columns}
  \column{0.5\textwidth}

  This is a text in first column.
  \[
    E=mc^{2}.
  \]

  \begin{itemize}
  \item First item
  \item Second item
  \end{itemize}

  \column{0.5\textwidth}
  This text will be in the second column and on a second thoughts,
  this is a nice looking layout in some cases.
\end{columns}
\end{frame}

\section{Теорема Пифагора}\label{sec:thm-pif}

\begin{transitionframe}
  \begin{center}
    { \Huge \textcolor{black}{Теорема Пифагора}}
  \end{center}
\end{transitionframe}


\begin{frame}{Прямоугольные треугольники}
  \onslide*<1>{
    \begin{definition}\label{def:triangle}
      \emph{Прямоугольным треугольником} называется треугольник, у
      которого один угол равен 90 градусов.
    \end{definition}
  }
    \begin{theorem}<2->
      Сумма квадратов катетов равна квадрату гипотенузы, т.е.
      \begin{equation}
        \label{eq:thm-pif}
        a^2+b^2=c^2.
      \end{equation}
    \end{theorem}
    \only<3>{
    \begin{proof}
      Доказательство очень простое.
    \end{proof}
    }
  \only<4>{
    \begin{remark}\label{rem:example:pif}
      Множество решений уравнения~\eqref{eq:thm-pif} называется
      пифагоровыми тройками.

      Обозначим их символом $\mathcal{C}$ (каллиграфическое $C$).

      Очевидно, что $\mathcal{C}\subseteq \mathbb{R}^{3}$.

      А это множество $\mathcal{B}$ (Каллиграфическое $B$).
    \end{remark}
  }
\end{frame}

\begin{frame}{Слайд с паузами (нажимайте далее)}
 In this slide \pause%

 the text will be partially visible \pause%

 And finally everything will be there.
\end{frame}

\begin{frame}{Окружения с нумерацией и без}
  \begin{proposition}\label{prop:example:test}
    Небольшое утверждение, которое ничего не постулирует, а просто
    показывается различные формулы, например, \(2^{x}\) или
    \(f(x)=x^{2}+ax+ b\).

    Также вот ещё одна формула
    \[
      \int\frac{e^{x}}{x^{2}}dx=Ei(x)-\frac{e^{x}}{x}+C.
    \]
    Конец утверждения.
  \end{proposition}

  \begin{proposition*}[Заметили? Нет номера]
    Это утверждение не имеет номера и содержит немного текста.
    % Некоторый текст в этом утверждении, он небольшой, но все же текст.
  \end{proposition*}
\end{frame}

\begin{frame}{Вставка картинок}
  \begin{columns}
    \column{0.5\textwidth}
    \begin{figure}[H]
      \centering
      \includegraphics[scale=0.2]{cmc-msu}
      \caption{Здание ВМК и ГЗ МГУ}\label{fig:example:cmc-gz}
    \end{figure}

    \column{0.5\textwidth}
    \begin{itemize}
    \item Картинка с главным зданием МГУ и корпусов факультета
      вычислительной математики и кибернетики.
    \item Очень красивые здания.
    \item Приезжайте на Воробьевы горы, чтобы их увидеть лично.
    \end{itemize}
    
  \end{columns}
\end{frame}

\section{Ссылки на разные научные статьи}\label{sec:ref-to-articles}

\begin{frame}{Ссылки}
  \begin{example}\label{ex:example:refs}
    \begin{itemize}
    \item Статья~\cite{ahlswede1977} посвящена геометрии булевого
      куба.
    \item Статья~\cite{ahmed2013} рассказывает о конструкции
      генератора псевдослучайных чисел на основе теории кодов,
      исправляющих ошибки.
    \end{itemize}
  \end{example}

  \begin{example*}
    Это пример без нумерации.
    Оформляется он также как и пример с нумерацией.
    В том числе изменяются цвета префикса списков:

    \begin{itemize}
    \item Один
    \item Два
    \end{itemize}
  \end{example*}
\end{frame}


\section{Гиперссылки}\label{sec:hyper}

\begin{frame}
  \frametitle{Ссылки}

  \begin{itemize}
  \item<1-> Ссылка на раздел~\ref{sec:thm-pif} с теоремой Пифагора.
  \item<2-> А формула~\eqref{eq:thm-pif} описывает суть теоремы
    Пифагора.
  \item<3-> Чтобы увидеть ссылки на разные научные статьи, смотри
    раздел \hyperref[sec:ref-to-articles]{\textquote{Ссылки на разные
        научные статьи}}.
  \item<2-> А вот ссылка на
    \href{https://ru.wikipedia.org}{википедию}.
  \end{itemize}
\end{frame}

\if \MINTED\empty
% do nothing, minted is off
\else
\section{Пакет minted}\label{sec:minted}

\begin{example}\label{ex:example:minted}
  Это пример кода на языке \emph{python}.
  \inputminted{python}{code.py}
\end{example}
\fi
\end{document}

%%% Local Variables:
%%% mode: latex
%%% TeX-master: t
%%% End:
